\chapter{Introduction}
\label{c:intro}

Computation and networking were first introduced in the 1970s out of the manufacturer's necessity to meet new and stringent emissions regulations \citep{pelkmans2003trends}. Then, the concept of connected cars appeared in Formula 1 in 1980, with the vehicle being able to transmit data to the pitlane \citep{bmwConnectedCar}. In 1996, General Motors introduced its OnStar roadside assistance program, where the vehicle could register an accident and automatically call the nearest emergency centre \citep{sspiConnectedCar}. The use of \gls{gps} by civillians was allowed in May of 2000, which not only improved navigation but also allowed stolen vehicles to be tracked \citep{bmwConnectedCar}. Remote diagnostics were introduced in 2001 and, since then, we saw the inclusion of vehicle health reports and 4G \gls{lte} networking. Manufacturers are currently working on adding 5G connectivity to their vehicles. This networking ability makes the modern vehicle a part of the \gls{iot}. Not only are vehicles more connected than ever, but they also have much more computing power. Many modern vehicles offer at least some driving assistance functionality and make decisions to avoid accidents.\par

This increase in functionality means that a modern car contains about 100 million lines of software code, and is expected to have around 300 million lines of code by 2030. For comparison, a passenger plane has around 15 million \citep{yahooCarSecurity}. This leaves a lot of room for vulnerabilities to appear, and many attacks have been successfully demonstrated. These include not only locking and unlocking the vehicle, and information theft, but also disruption of safety-critical components such as sensors and brakes \citep{Kim2021}. Manufacturers have, however, failed to keep up with the cybersecurity needs of the vehicles they produce. The two main reasons for this issue are that first, automakers specialised in the manufacturing of vehicles and not in software engineering, and second, there is an increased cost to making secure software \citep{forbesSecurityCost}.\par

Here, a prototype of an \gls{ids} for \gls{can}, which is currently the dominant networking protocol in the automotive industry, is presented. \glspl{ids} are common in traditional IP networks, but their implementation in \gls{can} is still being explored. Most proposed IDS methodologies are very resource heavy, with complex deep learning models often being implemented \citep{ahmad2021network}. This results in lower energy efficiency, which is of major importance to both the manufacturers and the customer. The proposed system can perform blind network traffic analysis and signal when an attack is occurring while being constrained by the limited computing resources that are present in a vehicle as well as its efficiency goals.