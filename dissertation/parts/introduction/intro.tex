\chapter{Introduction}
\label{c:intro}

Computation and networking were first introduced in the 1970s out of the manufacturer's necessity to meet new and stringent emissions regulations \citep{pelkmans2003trends}. Then, the concept of connected cars appeared in Formula 1 in 1980, with the vehicle being able to transmit data to the pitlane \citep{bmwConnectedCar}. In 1996, General Motors introduced its OnStar roadside assistance program, where the vehicle could register an accident and automatically call the nearest emergency centre \citep{sspiConnectedCar}. The use of \gls{gps} by civillians was allowed in May of 2000, which not only improved navigation but also allowed stolen vehicles to be tracked \citep{bmwConnectedCar}. Remote diagnostics were introduced in 2001 and, since then, we saw the inclusion of vehicle health reports and 4G \gls{lte} networking. Manufacturers are currently working on adding 5G connectivity to their vehicles. This networking ability makes the modern vehicle a part of the \gls{iot}. Not only are vehicles more connected than ever, but they also have much more computing power. Many modern vehicles offer at least some driving assistance functionality and make decisions to avoid accidents.\par

This increase in functionality means that a modern car contains about 100 million lines of software code, and is expected to have around 300 million lines of code by 2030. For comparison, a passenger plane has around 15 million \citep{yahooCarSecurity}. This leaves a lot of room for vulnerabilities to appear, and many attacks have been successfully demonstrated. These include not only locking and unlocking the vehicle, and information theft, but also disruption of safety-critical components such as sensors and brakes \citep{Kim2021}. Manufacturers have, however, failed to keep up with the cybersecurity needs of the vehicles they produce. The two main reasons for this issue are that first, automakers specialised in the manufacturing of vehicles and not in software engineering, and second, there is an increased cost to making secure software \citep{forbesSecurityCost}.\par

Here, a prototype of an \gls{ids} for \gls{can}, which is currently the dominant networking protocol in the automotive industry, is presented. \glspl{ids} are common in traditional IP networks, but their implementation in \gls{can} is still being explored. Most proposed IDS methodologies are very resource heavy, with complex deep learning models often being implemented \citep{ahmad2021network}. This results in lower energy efficiency, which is of major importance to both the manufacturers and the customer. The system here presented performs blind traffic analysis, meaning that it does not need to understand packet syntax. Since this is non-standardised and highly proprietary information, it allows the system to be installed in a variety of vehicles. It is also design to work in an embedded environment, constrained by the limited computing resources that are present in a vehicle as well as its efficiency goals.\par

This dissertation is organised as follows. In Chapter \ref{c:intro}, an overview of relevant concepts is first conducted. This includes modern vehicle software architectures, followed by automotive security standards, common attacks, and proposed solutions. Automotive in-vehicle networks are then detailed, before an in-depth look at the \acrlong{can} is taken. Relevant machine-learning concepts are then introduced.\par
A look into the state-of-the-art of \glspl{ids} is performed in Chapter \ref{c:sota}. It details modern detection approaches, along with evasion techniques, and goes over recent literature on the adaptation of the \gls{ids} concept into the \acrlong{can}. Automotive-specific challenges are also explored, followed by an approach to incident response.\par
% Chapter \ref{c:contribution} begins the dissertation's core content. It describes how this application is relevant and how it contributes to progress in the field of automotive cybersecurity.\par
Datasets used to develop and test the application are listed in Chapter \ref{c:datasets}. These vary from publicly available datasets to ones gathered from a real vehicle cluster for the purposes of this dissertation.\par
The proposed application is detailed in Chapter \ref{c:application}. The general architecture is first explained, followed by the training process and the features extracted from network traffic. The application's execution in an embedded environment, along with its options, concludes this chapter.\par
Test results are presented in Chapter \ref{c:results}, with Chapter \ref{c:limitations} presenting the application's limitations and detailing future work.\par
Chapter \ref{c:conclusion} concludes the dissertation.