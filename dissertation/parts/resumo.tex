\chapter*{Resumo}

% Os primeiros veículos possuíam muito poucos componentes eletrónicos. Isto tem vindo a mudar gradualmente, impulsionado principalmente pelas expectativas dos consumidores quanto ao aumento de conforto nos seus veículos, e pelos cada vez mais exigentes regulamentos de eficiência e emissões. Hoje, praticamente todas as funções do veículo são controladas eletrónicamente via Unidades de Controlo Eletrónico (UCEs) ligadas através várias tecnologias de transmissão tais como a Controller Area Network (CAN), a Local Interconnect Network (LIN), e a emergente Automotive Ethernet. Tais tecnologias aumentam o risco de ataques, especialmente agora que os veículos estão ligados ao exterior via Bluetooth, Wi-Fi, e rede móvel, e possuem informação sensível acerca do condutor. Redes tradicionais, como o IP, possuem tipicamente um Sistema de Deteção de Intrusão (SDI) que analiza o tráfego e sinaliza quando ocorre um ataque. O sistema aqui proposto é uma adaptação do SDI tradicional à CAN, uma vez que a CAN é a principal tecnologia de comunicação que liga a maior parte dos nodos, e não possui qualquer funcionalidade direcionada à segurança, como cifragem ou autenticação das mensagens. O SDI proposto utiliza uma One Class Support Vector Machine (OCSVM) treinada com dados extraídos de tráfego real, livre de ataques, para detetar com viabilidade uma variedade de ataques, tanto conhecidos como desconhecidos, sem a necessidade de entender a sintaxe do campo de dados das mensagens, que é maioritariamente proprietária. Isto permite ao sistema ser instalado em qualquer veículo num modo \textit{plug-and-play} enquanto mantém um elevado nível de desempenho com muito poucos falsos positivos.

A \gls{can} é a principal tecnologia de comunicação interna automóvel, ligando muitas \glspl{ecu} que controlam virtualmente todas as funções do veículo desde injeção de combustível até aos sensores de estacionamento. No entanto, não possui por defeito funcionalidades de segurança como cifragem ou autenticação. É possível aos atacantes facilmente injetarem ou modificarem pacotes na rede causando estragos e colocando em perigo tanto o condutor como os passageiros. Existe um número cada vez maior de \glspl{ecu} nos veículos modernos, impulsionado principalmente pelas expectativas do consumidores quanto ao aumento do conforto nos seus veículos, e pelos cada vez mais exigentes regulamentos de eficiência e emissões. Isto, associada à conexão ao exterior através de tecnologias como o Bluetooth, Wi-Fi, ou redes móveis, aumenta o risco de ataques. Redes tradicionais, como a rede \gls{ip}, tipicamente possuem um \glspl{ids} que analiza o tráfego e assinala a presença de um ataque. O sistema aqui proposto é uma adaptação do \gls{ids} tradicional à rede \gls{can} utilizando uma \gls{ocsvm} treinada com tráfego real e livre de ataques. O sistema é capaz de detetar com fiabilidade uma variedade de ataques, tanto conhecidos como desconhecidos, sem a necessidade de entender a sintaxe do campo de dados das mensagens, que é maioritariamente proprietária. Isto permite ao sistema ser instalado em qualquer veículo num modo \textit{plug-and-play} enquanto mantém um elevado nível de desempenho com muito poucos falsos positivos.

\paragraph{Palavras-chave} sistema de deteção de intrusão, aprendizagem máquina, controller area network, veículos conectados