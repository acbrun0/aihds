\chapter{Abstract}

% The first vehicles had very few, if any, electronic components. This has been changing gradually, primarily driven by the consumer's expectations of more features and comfort in their vehicles, and by ever-stricter government regulations on efficiency and emissions. Today, practically every function of the vehicle is controlled electronically via \glspl{ecu} connected by various bus technologies such as the \gls{can}, \gls{lin}, and the emerging Automotive Ethernet. Such technology raises the risk of attacks, especially now that vehicles are connected to the exterior via Bluetooth, Wi-Fi, and cellular, and hold sensitive information about the driver. Traditional networks, such as IP, typically have an \gls{ids} analysing traffic and signalling when an attack occurs. The system here proposed is an adaptation of the traditional IDS into the CAN bus, since CAN is the primary bus connecting most nodes, and its protocol does not have any security functionality such as encryption or message authentication. The proposed IDS uses a \gls{ocsvm} trained with features extracted from live, attack-free traffic, to reliably detect a variety of attacks, both known and unknown, without needing to understand the syntax of the packet's payload, which is largely proprietary. This allows the system to be installed in any vehicle in a plug-and-play fashion while maintaining a large degree of accuracy with very few false positives.

The \gls{can} in the backbone of automotive networking, connecting many \glspl{ecu} that control virtually every vehicle function from fuel injection to parking sensors. It possesses, however, no security functionality such as message encryption or authentication by default. Attackers can easily inject or modify packets in the network, causing vehicle malfunction and endangering the driver and passengers. There is an increasing number of \glspl{ecu} in modern vehicles, primarily driven by the consumer's expectation of more features and comfort in their vehicles as well as ever-stricter government regulations on efficiency and emissions. Combined with vehicle connectivity to the exterior via Bluetooth, Wi-Fi, or cellular, this raises the risk of attacks. Traditional networks, such as \gls{ip}, typically have an \gls{ids} analysing traffic and signalling when an attack occurs. The system here proposed is an adaptation of the traditional \gls{ids} into the \gls{can} bus using a \gls{ocsvm} trained with live, attack-free traffic. The system is capable of reliably detecting a variety of attacks, both known and unknown, without needing to understand payload syntax, which is largely proprietary and vehicle/model dependent. This allows it to be installed in any vehicle in a plug-and-play fashion while maintaining a large degree of accuracy with very few false positives.

\paragraph{Keywords} intrusion detection system, machine learning, controller area network, connected vehicles